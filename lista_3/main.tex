\documentclass[12pt,a4paper]{article}
\usepackage[utf8]{inputenc}
\usepackage[brazilian]{babel}
\usepackage{amsmath}
\usepackage{amsfonts}
\usepackage{amssymb}
\usepackage[T1]{fontenc}
\usepackage{graphicx}
\usepackage{physics}
\usepackage{siunitx}
\usepackage{listings}

\newcounter{prob}
\newcounter{subprob}
\renewcommand{\thesubprob}{\alph{subprob}}

\newcommand{\problem}[1]{\setcounter{subprob}{0} \stepcounter{prob} \par \medskip \noindent \textbf{#1 \ .}}

\newcommand{\answer}{\par \medskip \noindent \textit{\textbf{prova} \ }}

\newcommand{\finalanswer}[1]{
	\begin{center} 
    	{\renewcommand{\arraystretch}{1.5}
		\renewcommand{\tabcolsep}{0.2cm} 
    	\begin{tabular}{|c|} 
    		\hline 
        	$ \displaystyle #1 $  \\ 
        	\hline 
    	\end{tabular}} 
   	\end{center}}

\newcommand{\subproblem}[1]{ \par \smallskip \noindent \quad \textit{(#1)  \ }}

\newcommand{\subanswer}{\par \smallskip \noindent \quad \textit{ \ }}

\newcommand{\option}{\item[$\square$]}
\newcommand{\thisone}{\item[$\blacksquare$]}

\newenvironment{subitemize}{\begin{itemize}}{\end{itemize}}


\author{Leonardo Mendes de Moraes }
\title{Lista 3 - MAC5711}
\date{}
\begin{document}

%%--CABEÇALHO--%%
	\begin{center}
    {\huge Lista 3 \par} {\LARGE Análise de Algoritmos \par} {\Large MAC5711
    \par}
	\end{center}

\problem{1}
    \begin{lstlisting}
Algoritmo MaiorMenor (v, n)
    maior <- v[1]
    menor <- v[1]
    para i <- 2 ate n faca
        se v[i] > maior
            entao maior <- v[i]
            senao se v[i] < menor
                entao menor <- v[i]
    devolva maior, menor
        \end{lstlisting}
    \subproblem{a} Qual é o número esperado de comparações executadas na linha 6
    do algoritmo?

    A linha \verb|6| somente é executada se a primeira condição da linha
    \verb|4| não for verdadeira. Portanto, para saber quantas vezes a linha
    \verb|6| é executada, deve ser calculada a quantidade de vezes que a linha
    \verb|5| é executada primemiro.

    Considere que: $X(v, n) = \text{número de execuções da linha 5}$.

    \begin{equation}
        E[X] = \Sigma_{s \in S} X(s) \cdot Pr(s) \\
        \notag
    \end{equation}
    \begin{equation}
        X_i =
            \begin{cases}
                1 & \text{se $v[i]$ > maior}\\
                0 & \text{caso contrário}
            \end{cases}
            \notag
    \end{equation}
    \begin{multline}
        \therefore E[X_i] = \text{probabilidade que $A[i]$ seja o valor máximo em $A[1...i]$}
        = \frac{1}{i} \\
        \begin{split}
        E[X] &= E[X_i + ... + X_n] = E[X_1] + .. + E[X_n] = \frac{1}{1} + ... + \frac{1}{n} \\
            &< 1 + \ln n = \Theta(\lg n)
    \end{split}
    \notag
    \end{multline}
    Achado o número de execuções da linha \verb|5|, para achar o da linha
    \verb|6| é o número de vezes que o laço é executado menos o número de
    execuções da \verb|5|. Se o laço executa \emph{n} vezes e a linha \verb|5|
    executa $\Theta(\lg n)$ vezes, pode-se concluir que o número de execuções da
    linha \verb|6| é:
    \begin{equation}
    \Theta(n) -\Theta(\lg n) = \Theta(n - \lg n)
    \notag
    \end{equation}

    \subproblem{b} Qual é o número esperado de atribuições efetuadas na linha 7 do algoritmo?
    \begin{multline}
        X(v, n) = \text{número de execuções da linha 7} \\
        E[X] = \Sigma_{s \in S} X(s) \cdot Pr(s) \\
        \notag
    \end{multline}
    Dividindo o problema em $X_i$:
    \begin{equation}
        X_i =
            \begin{cases}
                1 & \text{se $v[i] < maior$ e $v[i] < menor$} \\
                0 & \text{caso não.} \\
            \end{cases}
    \notag
    \end{equation}
    O $v[i] < maior$ é o inverso do cálculo da linha \verb|5|. Pois para $X_i$
    ser executado, é necessário que não entre na condição da linha \verb|4| e
    que a condição da linha \verb|6| seja verdadeira. Portanto:
    \begin{multline}
        X_i = \frac{i-1}{i} \cdot \frac{1}{i} = \frac{i-1}{i^2} = \frac{1}{i} - \frac{1}{i^2} \\
        \begin{split}
        E[X] &= (\frac{1}{1} - \frac{1}{1} ) + (\frac{1}{2} - \frac{1}{4}) + (\frac{1}{3} - \frac{1}{9}) + ... + (\frac{1}{n} - \frac{1}{n^2}) \\
            & = \Sigma_{i=1}^n \frac{1}{i} + {\Sigma_{i=1}^n \frac{-1}{i^2} }
        \end{split}
        \notag
    \end{multline}
    As duas somatórias podem ser consideradas séries harmônicas. Resolvendo a
    primeira, como visto anteriormente:
    \begin{equation}
        \Sigma_{i=1}^n \frac{1}{i} < \ln + 1 = \Theta(\lg n)
        \notag
    \end{equation}
    A segunda somatória é uma série que pode ser simplificada por:
    \begin{equation}
        \begin{split}
        \Sigma_{i=1}^n \frac{-1}{i^2} &= \frac{-1}{1} + \frac{-1}{4} - \frac{-1}{9} + \frac{-1}{16} + .. + \frac{-1}{n^2} \\
            &< \underbrace{1 - \frac{1}{2} + \frac{1}{3} - \frac{1}{4} + \frac{1}{5} + ... }_{\text{Série Hamônica alternada} } < \ln 2 \\
        \end{split}
        \notag
    \end{equation}
    A série harmônica alternada converge em um valor, portanto:
    \begin{equation}
        \Sigma_{i=1}^n \frac{1}{i} + \Sigma_{i=1}^n \frac{-1}{i^2} = \theta(\lg n)
        \notag
    \end{equation}
    Então, o número esperado de atribuições é $\Theta(\lg n)$ para a linha
\verb|7|. \problem{3} Para esta questão, vamos dizer que a mediana de um vetor
A[p . . r] com número inteiros é o valor que ficaria na posição $A[\lfloor(p + r)/2\rfloor]$
depois que o vetor $A[p . . r]$ fosse ordenado. Dado um algoritmo linear
“caixa-preta” que devolve a mediana de um vetor, descreva um algoritmos, linear,
que, dado um vetor $A[p . . r]$ de inteiros distintos e um inteiro k, devolve o
k-ésimo menor do vetor. (O k-ésimo menor de um vetor de inteiros distintos é o
elemento que estaria na k-ésima posição do vetor se ele fosse ordenado). Você
pode assumir que o vetor tem tamanho igual uma potência de 2.
\begin{lstlisting}[numbers=left,
    stepnumber=1,]
kEsimoTermo(A, p, r, k):
    valor_mediana = mediana(A, p, r)
    esquerda = [elementos de A <= valor_mediana]
    direita = [elementos de A >= valor_mediana]
    k_Atual = len(esquerda)
    se k < k_Atual:
        retorna kEsimoTermo(esquerda, 0, k_Atual - 1, k)
        senao se k > k_Atual:
            retorna kEsimoTermo(direita, 0, len(direita), k - k_Atual)
    retorna valor_mediana
\end{lstlisting}
O algoritmo funciona da seguinte maneira: obtem o valor da mediana. A partir
dela, cria-se uma lista com todos os valores menores que a mediana e uma outra
lista com todos os valores maiores. Com isso, é possível saber a posição exata
da mediana e comparar o valor de $k$ recebido na função com o tamanho das duas
listas.

Se \verb|k| for menor que o índice atual da mediana significa que o
\emph{k-Ésimo} elemento está na lista da esquerda. Chama e retorna recursivamente a função
passando a lista dos valores menores que a mediana.

Se \verb|k| for maior que o índice atual da mediana significa que o seu elemento está
na lista dos elementos maiores que a mediana, portanto retorna o valor obtido da chamada da função
recursivamente com o vetor dos valores maiores. No entanto, temos que atualizar
o valor do arguemtno \verb|k|, subtraindo o número total de elementos que serão
ignorados da lista \verb|esquerda|. Por fim, se \verb|k| não é nem maior nem menor que \verb|k_Atual| significa que
encontramos o \emph{k-ésimo} valor, portanto o valor da mediana do vetor atual é
retornado.

Considerando $T(n)$ como a função de tempo do algoritmo, temos:


\begin{lstlisting}[numbers=left,
    stepnumber=1,escapeinside={(*}{*)}]
    (*$\Theta(N)$*)
    (*$\Theta(N)$*)
    (*$\Theta(N)$*)
    (*$\Theta(1)$*)
    (*$\Theta(1)$*)
    (*$T(\frac{n}{2})$*)
    (*$T(\frac{n}{2})$*)
    (*$\Theta(1)$*)

\end{lstlisting}

Portanto, pode-se definir a função $T(n)$ como:
\begin{equation}
    T(n) = T(\frac{n}{2}) + n
    \notag
\end{equation}
\begin{equation}
    \begin{split}
        T(n) &= T(\frac{n}{2}) + n \\
        &= T(\frac{n}{4}) + n(1+\frac{1}{2}) \\
        &= T(\frac{n}{8}) + n(1 +\frac{1}{2} + \frac{1}{4}) \\
        &\vdots\\
        &= T(\frac{n}{2^k}) + n\underbrace{(1+\frac{1}{2}+\frac{1}{4} + ... + \frac{1}{2^{k-1}})}_{\text{Soma de P.G.}}
    \end{split}
    \notag
\end{equation}
Igualando $k = \lg n$, temos:
\begin{equation}
    \begin{split}
    T(n) &= T(1) + n(1+\frac{1}{2}+\frac{1}{4} + ... + \frac{1}{2^{\lg n -1}}) \\
        &= 1+ 2n - 2 \\
        &= \Theta(n)
    \end{split}
    \notag
\end{equation}
\end{document}
