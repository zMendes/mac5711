\documentclass[12pt,a4paper]{article}
\usepackage[utf8]{inputenc}
\usepackage[brazilian]{babel}
\usepackage{amsmath}
\usepackage{amsfonts}
\usepackage{amssymb}
\usepackage[T1]{fontenc}
\usepackage{graphicx}
\usepackage{physics}
\usepackage{siunitx}
\newcounter{prob}
\newcounter{subprob}
\renewcommand{\thesubprob}{\alph{subprob}}

\newcommand{\problem}[1]{\setcounter{subprob}{0} \stepcounter{prob} \par \medskip \noindent \textbf{#1 \ .}}

\newcommand{\answer}{\par \medskip \noindent \textit{\textbf{prova} \ }}

\newcommand{\finalanswer}[1]{
	\begin{center} 
    	{\renewcommand{\arraystretch}{1.5}
		\renewcommand{\tabcolsep}{0.2cm} 
    	\begin{tabular}{|c|} 
    		\hline 
        	$ \displaystyle #1 $  \\ 
        	\hline 
    	\end{tabular}} 
   	\end{center}}

\newcommand{\subproblem}[1]{ \par \smallskip \noindent \quad \textit{(#1)  \ }}

\newcommand{\subanswer}{\par \smallskip \noindent \quad \textit{ \ }}

\newcommand{\option}{\item[$\square$]}
\newcommand{\thisone}{\item[$\blacksquare$]}

\newenvironment{subitemize}{\begin{itemize}}{\end{itemize}}

\author{Leonardo Mendes de Moraes }
\title{Lista de Exercícios - MAC5711}
\date{}
\begin{document}

%%--CABEÇALHO--%%
	\begin{center}
    {\huge Lista 1 \par} {\LARGE Análise de Algoritmos \par} {\Large MAC5711
    \par}
	\end{center}

\problem
    \subproblem{b} $\log_{10} n$ é O($\lg n$) 
        \answer Para $n \geq 1$ e $c_1 =
        \log_{10} 2$ temos que: \\
        $\log_{10} n \leq \log_{10}2
        \lg n$ \\
        $\log_{10} n \leq \log_{10}2 . \frac{\log_{10}n}{\log_{10}2}$ \\
        $\log_{10} n \leq \log_{10}n$.


\problem
	\subproblem{d} $n = O(2^n)$
        \answer Para todo $n_0 = 0$ e $c_1 = 1$ onde $n \geq n_0$, temos que
        $0 \leq n \leq 2^n$.
\problem
    \subproblem{b} Se $f(n) =  \Theta(g(n))$ e $g(n) = \Theta(h(n))$ então $f(n)
    = \Theta(h(n))$.
        \answer Temos que $f(n) = \Theta(g(n))$ quando $f(n) =
        O(g(n))$ e $f(n) = \Omega(g(n))$:
        \begin{equation}
            c_1 g(n) \leq f(n) \leq c_2 g(n)
        \end{equation}
        E $g(n) = \Theta(h(n))$ quando $g(n) =
        O(h(n))$ e $g(n) = \Omega(h(n))$
        \begin{equation}
            c_3 h(n) \leq g(n) \leq c_4 h(n)
        \end{equation}
        $f(n) = \Theta(h(n))$ é verdadeiro pois se (1) e (2), existe um $c_1$, $c_2$, $c_3$ e $c_4$ em que temos: 
        \begin{equation}
            c_3 h(n) \leq  c_2 g(n) \leq f(n) \leq  c_2 g(n) \leq c_4 h(n) \nonumber
        \end{equation}
        \begin{equation}
            c_3 h(n) \leq f(n) \leq c_4 h(n) \nonumber
        \end{equation}
        Logo, $f(n)$ é $\Theta(h(n))$.

\end{document}